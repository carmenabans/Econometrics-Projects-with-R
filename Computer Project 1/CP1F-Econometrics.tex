% Options for packages loaded elsewhere
\PassOptionsToPackage{unicode}{hyperref}
\PassOptionsToPackage{hyphens}{url}
%
\documentclass[
]{article}
\usepackage{amsmath,amssymb}
\usepackage{lmodern}
\usepackage{ifxetex,ifluatex}
\ifnum 0\ifxetex 1\fi\ifluatex 1\fi=0 % if pdftex
  \usepackage[T1]{fontenc}
  \usepackage[utf8]{inputenc}
  \usepackage{textcomp} % provide euro and other symbols
\else % if luatex or xetex
  \usepackage{unicode-math}
  \defaultfontfeatures{Scale=MatchLowercase}
  \defaultfontfeatures[\rmfamily]{Ligatures=TeX,Scale=1}
\fi
% Use upquote if available, for straight quotes in verbatim environments
\IfFileExists{upquote.sty}{\usepackage{upquote}}{}
\IfFileExists{microtype.sty}{% use microtype if available
  \usepackage[]{microtype}
  \UseMicrotypeSet[protrusion]{basicmath} % disable protrusion for tt fonts
}{}
\makeatletter
\@ifundefined{KOMAClassName}{% if non-KOMA class
  \IfFileExists{parskip.sty}{%
    \usepackage{parskip}
  }{% else
    \setlength{\parindent}{0pt}
    \setlength{\parskip}{6pt plus 2pt minus 1pt}}
}{% if KOMA class
  \KOMAoptions{parskip=half}}
\makeatother
\usepackage{xcolor}
\IfFileExists{xurl.sty}{\usepackage{xurl}}{} % add URL line breaks if available
\IfFileExists{bookmark.sty}{\usepackage{bookmark}}{\usepackage{hyperref}}
\hypersetup{
  pdftitle={Computer Proyect 1},
  pdfauthor={Carmen Abans},
  hidelinks,
  pdfcreator={LaTeX via pandoc}}
\urlstyle{same} % disable monospaced font for URLs
\usepackage[margin=1in]{geometry}
\usepackage{color}
\usepackage{fancyvrb}
\newcommand{\VerbBar}{|}
\newcommand{\VERB}{\Verb[commandchars=\\\{\}]}
\DefineVerbatimEnvironment{Highlighting}{Verbatim}{commandchars=\\\{\}}
% Add ',fontsize=\small' for more characters per line
\usepackage{framed}
\definecolor{shadecolor}{RGB}{248,248,248}
\newenvironment{Shaded}{\begin{snugshade}}{\end{snugshade}}
\newcommand{\AlertTok}[1]{\textcolor[rgb]{0.94,0.16,0.16}{#1}}
\newcommand{\AnnotationTok}[1]{\textcolor[rgb]{0.56,0.35,0.01}{\textbf{\textit{#1}}}}
\newcommand{\AttributeTok}[1]{\textcolor[rgb]{0.77,0.63,0.00}{#1}}
\newcommand{\BaseNTok}[1]{\textcolor[rgb]{0.00,0.00,0.81}{#1}}
\newcommand{\BuiltInTok}[1]{#1}
\newcommand{\CharTok}[1]{\textcolor[rgb]{0.31,0.60,0.02}{#1}}
\newcommand{\CommentTok}[1]{\textcolor[rgb]{0.56,0.35,0.01}{\textit{#1}}}
\newcommand{\CommentVarTok}[1]{\textcolor[rgb]{0.56,0.35,0.01}{\textbf{\textit{#1}}}}
\newcommand{\ConstantTok}[1]{\textcolor[rgb]{0.00,0.00,0.00}{#1}}
\newcommand{\ControlFlowTok}[1]{\textcolor[rgb]{0.13,0.29,0.53}{\textbf{#1}}}
\newcommand{\DataTypeTok}[1]{\textcolor[rgb]{0.13,0.29,0.53}{#1}}
\newcommand{\DecValTok}[1]{\textcolor[rgb]{0.00,0.00,0.81}{#1}}
\newcommand{\DocumentationTok}[1]{\textcolor[rgb]{0.56,0.35,0.01}{\textbf{\textit{#1}}}}
\newcommand{\ErrorTok}[1]{\textcolor[rgb]{0.64,0.00,0.00}{\textbf{#1}}}
\newcommand{\ExtensionTok}[1]{#1}
\newcommand{\FloatTok}[1]{\textcolor[rgb]{0.00,0.00,0.81}{#1}}
\newcommand{\FunctionTok}[1]{\textcolor[rgb]{0.00,0.00,0.00}{#1}}
\newcommand{\ImportTok}[1]{#1}
\newcommand{\InformationTok}[1]{\textcolor[rgb]{0.56,0.35,0.01}{\textbf{\textit{#1}}}}
\newcommand{\KeywordTok}[1]{\textcolor[rgb]{0.13,0.29,0.53}{\textbf{#1}}}
\newcommand{\NormalTok}[1]{#1}
\newcommand{\OperatorTok}[1]{\textcolor[rgb]{0.81,0.36,0.00}{\textbf{#1}}}
\newcommand{\OtherTok}[1]{\textcolor[rgb]{0.56,0.35,0.01}{#1}}
\newcommand{\PreprocessorTok}[1]{\textcolor[rgb]{0.56,0.35,0.01}{\textit{#1}}}
\newcommand{\RegionMarkerTok}[1]{#1}
\newcommand{\SpecialCharTok}[1]{\textcolor[rgb]{0.00,0.00,0.00}{#1}}
\newcommand{\SpecialStringTok}[1]{\textcolor[rgb]{0.31,0.60,0.02}{#1}}
\newcommand{\StringTok}[1]{\textcolor[rgb]{0.31,0.60,0.02}{#1}}
\newcommand{\VariableTok}[1]{\textcolor[rgb]{0.00,0.00,0.00}{#1}}
\newcommand{\VerbatimStringTok}[1]{\textcolor[rgb]{0.31,0.60,0.02}{#1}}
\newcommand{\WarningTok}[1]{\textcolor[rgb]{0.56,0.35,0.01}{\textbf{\textit{#1}}}}
\usepackage{graphicx}
\makeatletter
\def\maxwidth{\ifdim\Gin@nat@width>\linewidth\linewidth\else\Gin@nat@width\fi}
\def\maxheight{\ifdim\Gin@nat@height>\textheight\textheight\else\Gin@nat@height\fi}
\makeatother
% Scale images if necessary, so that they will not overflow the page
% margins by default, and it is still possible to overwrite the defaults
% using explicit options in \includegraphics[width, height, ...]{}
\setkeys{Gin}{width=\maxwidth,height=\maxheight,keepaspectratio}
% Set default figure placement to htbp
\makeatletter
\def\fps@figure{htbp}
\makeatother
\setlength{\emergencystretch}{3em} % prevent overfull lines
\providecommand{\tightlist}{%
  \setlength{\itemsep}{0pt}\setlength{\parskip}{0pt}}
\setcounter{secnumdepth}{-\maxdimen} % remove section numbering
\ifluatex
  \usepackage{selnolig}  % disable illegal ligatures
\fi

\title{Computer Proyect 1}
\author{Carmen Abans}
\date{2021-10-04}

\begin{document}
\maketitle

\begin{enumerate}
\def\labelenumi{\alph{enumi})}
\tightlist
\item
  Use the read.cvs command to read the Earnings\_and\_Height.cvs data
  set into R. Use the attach command to attach the data set into R.
\end{enumerate}

\begin{Shaded}
\begin{Highlighting}[]
\NormalTok{eah }\OtherTok{\textless{}{-}} \FunctionTok{read.csv}\NormalTok{(}\StringTok{"Earnings\_and\_Height.csv"}\NormalTok{)}
\FunctionTok{attach}\NormalTok{(eah)}
\end{Highlighting}
\end{Shaded}

\begin{enumerate}
\def\labelenumi{\alph{enumi})}
\setcounter{enumi}{1}
\tightlist
\item
  Print out an summary of the data set. In particular, find and report
  the sample average of the variables earnings, height and sex,
  respectively.
\end{enumerate}

\begin{Shaded}
\begin{Highlighting}[]
\FunctionTok{summary}\NormalTok{(eah)      }\CommentTok{\# The summary shows the following information about the variables in eah data set: Min., 1st Qu., Median, Mean, 3rd Qu., Max}
\end{Highlighting}
\end{Shaded}

\begin{verbatim}
##       sex              age             mrd             educ      
##  Min.   :0.0000   Min.   :25.00   Min.   :1.000   Min.   : 0.00  
##  1st Qu.:0.0000   1st Qu.:33.00   1st Qu.:1.000   1st Qu.:12.00  
##  Median :0.0000   Median :40.00   Median :1.000   Median :13.00  
##  Mean   :0.4419   Mean   :40.92   Mean   :2.362   Mean   :13.54  
##  3rd Qu.:1.0000   3rd Qu.:48.00   3rd Qu.:4.000   3rd Qu.:16.00  
##  Max.   :1.0000   Max.   :65.00   Max.   :6.000   Max.   :19.00  
##     cworker          region           race          earnings    
##  Min.   :1.000   Min.   :1.000   Min.   :1.000   Min.   : 4726  
##  1st Qu.:1.000   1st Qu.:2.000   1st Qu.:1.000   1st Qu.:23363  
##  Median :1.000   Median :3.000   Median :1.000   Median :38925  
##  Mean   :1.964   Mean   :2.551   Mean   :1.386   Mean   :46875  
##  3rd Qu.:3.000   3rd Qu.:3.000   3rd Qu.:1.000   3rd Qu.:84055  
##  Max.   :6.000   Max.   :4.000   Max.   :4.000   Max.   :84055  
##      height          weight        occupation    
##  Min.   :48.00   Min.   : 80.0   Min.   : 1.000  
##  1st Qu.:64.00   1st Qu.:140.0   1st Qu.: 2.000  
##  Median :67.00   Median :163.0   Median : 5.000  
##  Mean   :66.96   Mean   :170.4   Mean   : 6.011  
##  3rd Qu.:70.00   3rd Qu.:190.0   3rd Qu.: 8.000  
##  Max.   :84.00   Max.   :501.0   Max.   :15.000
\end{verbatim}

\begin{Shaded}
\begin{Highlighting}[]
\FunctionTok{summary}\NormalTok{(earnings) }
\end{Highlighting}
\end{Shaded}

\begin{verbatim}
##    Min. 1st Qu.  Median    Mean 3rd Qu.    Max. 
##    4726   23363   38925   46875   84055   84055
\end{verbatim}

\begin{Shaded}
\begin{Highlighting}[]
\FunctionTok{summary}\NormalTok{(height)       }
\end{Highlighting}
\end{Shaded}

\begin{verbatim}
##    Min. 1st Qu.  Median    Mean 3rd Qu.    Max. 
##   48.00   64.00   67.00   66.96   70.00   84.00
\end{verbatim}

\begin{Shaded}
\begin{Highlighting}[]
\FunctionTok{summary}\NormalTok{(sex)      }\CommentTok{\# The sex is a dummy variable where: 1=Male, 0=Female}
\end{Highlighting}
\end{Shaded}

\begin{verbatim}
##    Min. 1st Qu.  Median    Mean 3rd Qu.    Max. 
##  0.0000  0.0000  0.0000  0.4419  1.0000  1.0000
\end{verbatim}

\begin{enumerate}
\def\labelenumi{\alph{enumi})}
\setcounter{enumi}{2}
\tightlist
\item
  Run a regression of earnings on height. In particular, find and use a
  sentence to interpret the meaning of the regression coefficient of the
  variables height.
\end{enumerate}

\begin{Shaded}
\begin{Highlighting}[]
\CommentTok{\# Regression}
\NormalTok{ols }\OtherTok{\textless{}{-}} \FunctionTok{lm}\NormalTok{(earnings }\SpecialCharTok{\textasciitilde{}}\NormalTok{ height)}
\FunctionTok{summary}\NormalTok{(ols)}
\end{Highlighting}
\end{Shaded}

\begin{verbatim}
## 
## Call:
## lm(formula = earnings ~ height)
## 
## Residuals:
##    Min     1Q Median     3Q    Max 
## -47836 -21879  -7976  34323  50599 
## 
## Coefficients:
##             Estimate Std. Error t value Pr(>|t|)    
## (Intercept)  -512.73    3386.86  -0.151     0.88    
## height        707.67      50.49  14.016   <2e-16 ***
## ---
## Signif. codes:  0 '***' 0.001 '**' 0.01 '*' 0.05 '.' 0.1 ' ' 1
## 
## Residual standard error: 26780 on 17868 degrees of freedom
## Multiple R-squared:  0.01088,    Adjusted R-squared:  0.01082 
## F-statistic: 196.5 on 1 and 17868 DF,  p-value: < 2.2e-16
\end{verbatim}

\begin{Shaded}
\begin{Highlighting}[]
\CommentTok{\# We have got that the height has this coefficients:}

\CommentTok{\#   Estimate    707.67      (β1) This means that when the height increases by 1 (one inch taller) the earnings increase by $707.67.}
\CommentTok{\#                           the earnings increase by $707.67.}

\CommentTok{\#   Std. Error  50.49       (standard error of β1) This means that the average distance that the observed values}
\CommentTok{\#                           deviate from the regression line is 50.49 }
\CommentTok{\#                           (The smaller the value, the closer our values are to the regression line)}

\CommentTok{\#   t value     14.016      This is the coefficient divided by its standard error}

\CommentTok{\#   Pr(\textgreater{}|t|)    \textless{}2e{-}16 ***  p{-}value}
\end{Highlighting}
\end{Shaded}

\begin{enumerate}
\def\labelenumi{\alph{enumi})}
\setcounter{enumi}{3}
\tightlist
\item
  Plot a graph of earnings over height. e) On the graph, add a fitted
  line of the regression.
\end{enumerate}

\begin{Shaded}
\begin{Highlighting}[]
\FunctionTok{plot}\NormalTok{(earnings }\SpecialCharTok{\textasciitilde{}}\NormalTok{ height)     }\CommentTok{\# Graph}
\FunctionTok{abline}\NormalTok{(ols)                 }\CommentTok{\# Fitted line}
\end{Highlighting}
\end{Shaded}

\includegraphics{CP1F-Econometrics_files/figure-latex/unnamed-chunk-4-1.pdf}

\begin{enumerate}
\def\labelenumi{\alph{enumi})}
\setcounter{enumi}{5}
\tightlist
\item
  Suppose Alex is 65 inches; Bob is 67 inches; Chris is 70 inches tall.
  Based on the regression, predict their corresponding earnings.
\end{enumerate}

\begin{Shaded}
\begin{Highlighting}[]
\CommentTok{\# Alex (Method 1)}
\SpecialCharTok{{-}}\FloatTok{512.73} \SpecialCharTok{+} \FloatTok{707.67}\SpecialCharTok{*}\DecValTok{65}
\end{Highlighting}
\end{Shaded}

\begin{verbatim}
## [1] 45485.82
\end{verbatim}

\begin{Shaded}
\begin{Highlighting}[]
\CommentTok{\# Bob (Method 2)}
\NormalTok{ols}\SpecialCharTok{$}\NormalTok{coefficient[}\DecValTok{1}\NormalTok{] }\SpecialCharTok{+}\NormalTok{ ols}\SpecialCharTok{$}\NormalTok{coefficient[}\DecValTok{2}\NormalTok{] }\SpecialCharTok{*} \DecValTok{67}
\end{Highlighting}
\end{Shaded}

\begin{verbatim}
## (Intercept) 
##    46901.26
\end{verbatim}

\begin{Shaded}
\begin{Highlighting}[]
\CommentTok{\# Chris(Method 3)}
\FunctionTok{predict}\NormalTok{(ols, }\FunctionTok{data.frame}\NormalTok{(}\AttributeTok{height=}\DecValTok{70}\NormalTok{))}
\end{Highlighting}
\end{Shaded}

\begin{verbatim}
##        1 
## 49024.28
\end{verbatim}

\begin{enumerate}
\def\labelenumi{\alph{enumi})}
\setcounter{enumi}{6}
\tightlist
\item
  Find the R2 and SER from the regression in part (c). Use a sentence to
  interpret each of them.
\end{enumerate}

\begin{Shaded}
\begin{Highlighting}[]
\FunctionTok{summary}\NormalTok{(ols)}
\end{Highlighting}
\end{Shaded}

\begin{verbatim}
## 
## Call:
## lm(formula = earnings ~ height)
## 
## Residuals:
##    Min     1Q Median     3Q    Max 
## -47836 -21879  -7976  34323  50599 
## 
## Coefficients:
##             Estimate Std. Error t value Pr(>|t|)    
## (Intercept)  -512.73    3386.86  -0.151     0.88    
## height        707.67      50.49  14.016   <2e-16 ***
## ---
## Signif. codes:  0 '***' 0.001 '**' 0.01 '*' 0.05 '.' 0.1 ' ' 1
## 
## Residual standard error: 26780 on 17868 degrees of freedom
## Multiple R-squared:  0.01088,    Adjusted R-squared:  0.01082 
## F-statistic: 196.5 on 1 and 17868 DF,  p-value: < 2.2e-16
\end{verbatim}

\begin{Shaded}
\begin{Highlighting}[]
\CommentTok{\#     When we did the summary of the ols we got a residual standard error (SER) of $26780 and it is the measure}
\CommentTok{\#     of the spread of the error term u.}

\CommentTok{\#     We also got the R2 which it appears to be 0.01088. This means that approximately 1.08\% of earnings are }
\CommentTok{\#     explained by the height.}
\end{Highlighting}
\end{Shaded}

\begin{enumerate}
\def\labelenumi{\alph{enumi})}
\setcounter{enumi}{7}
\tightlist
\item
  Based on the regression in part (c), find the p-value of the variables
  height and perform a t-test.
\end{enumerate}

\begin{Shaded}
\begin{Highlighting}[]
\FunctionTok{summary}\NormalTok{(ols)}
\end{Highlighting}
\end{Shaded}

\begin{verbatim}
## 
## Call:
## lm(formula = earnings ~ height)
## 
## Residuals:
##    Min     1Q Median     3Q    Max 
## -47836 -21879  -7976  34323  50599 
## 
## Coefficients:
##             Estimate Std. Error t value Pr(>|t|)    
## (Intercept)  -512.73    3386.86  -0.151     0.88    
## height        707.67      50.49  14.016   <2e-16 ***
## ---
## Signif. codes:  0 '***' 0.001 '**' 0.01 '*' 0.05 '.' 0.1 ' ' 1
## 
## Residual standard error: 26780 on 17868 degrees of freedom
## Multiple R-squared:  0.01088,    Adjusted R-squared:  0.01082 
## F-statistic: 196.5 on 1 and 17868 DF,  p-value: < 2.2e-16
\end{verbatim}

\begin{Shaded}
\begin{Highlighting}[]
\CommentTok{\#     The summary of the ols shows that our p{-}value is smaller than 2.2e{-}16. }
\CommentTok{\#     If we do the t{-}test based on the p{-}value we get that the absolute value of the p{-}value is smaller than 1.96 }
\CommentTok{\#     For that reason we reject the null (H0: β1=0). }
\CommentTok{\#     That means that β1!=0 so there is a relationship between height and earnings.}
\end{Highlighting}
\end{Shaded}

\begin{enumerate}
\def\labelenumi{\roman{enumi})}
\tightlist
\item
  Based on the regression in part (c), use the confint command to
  calculate the Confidence Interval (CI) of the variables height. Does
  your CI give you the same t-test conclusion?
\end{enumerate}

\begin{Shaded}
\begin{Highlighting}[]
\FunctionTok{confint}\NormalTok{(ols)}
\end{Highlighting}
\end{Shaded}

\begin{verbatim}
##                  2.5 %    97.5 %
## (Intercept) -7151.2994 6125.8322
## height        608.7078  806.6353
\end{verbatim}

\begin{Shaded}
\begin{Highlighting}[]
\CommentTok{\# We\textquotesingle{}ve got that the CI = [608.7078, 806.6353]. Since 0 is out of the CI we also end up rejecting the null.}
\CommentTok{\# This is to be expected as all three t{-}test methods are equivalent.}
\end{Highlighting}
\end{Shaded}

\begin{enumerate}
\def\labelenumi{\alph{enumi})}
\setcounter{enumi}{9}
\tightlist
\item
  Run a regression of earnings on sex. For both the regression intercept
  and coefficient of the variables sex, use a sentence to interpret its
  meaning.
\end{enumerate}

\begin{Shaded}
\begin{Highlighting}[]
\NormalTok{ols2}\OtherTok{\textless{}{-}} \FunctionTok{lm}\NormalTok{(earnings }\SpecialCharTok{\textasciitilde{}}\NormalTok{ sex)}
\FunctionTok{summary}\NormalTok{(ols2)}
\end{Highlighting}
\end{Shaded}

\begin{verbatim}
## 
## Call:
## lm(formula = earnings ~ sex)
## 
## Residuals:
##    Min     1Q Median     3Q    Max 
## -43733 -22258  -6696  35595  38434 
## 
## Coefficients:
##             Estimate Std. Error t value Pr(>|t|)    
## (Intercept)  45621.0      269.2 169.455  < 2e-16 ***
## sex           2838.8      405.0   7.009 2.49e-12 ***
## ---
## Signif. codes:  0 '***' 0.001 '**' 0.01 '*' 0.05 '.' 0.1 ' ' 1
## 
## Residual standard error: 26890 on 17868 degrees of freedom
## Multiple R-squared:  0.002742,   Adjusted R-squared:  0.002686 
## F-statistic: 49.13 on 1 and 17868 DF,  p-value: 2.485e-12
\end{verbatim}

\begin{Shaded}
\begin{Highlighting}[]
\CommentTok{\# We\textquotesingle{}ve got that Earnings = 45621 + 2838.8 x Male (as the sex variables are 1=Male, 0=Female).}
\CommentTok{\# This means that men earn on average $2838.8 more than women.}
\CommentTok{\# We can also see that the mean earnings of women is $45621.}
\CommentTok{\# And if we want to know the mean earnings of men we just need to set the Male=1 which results in:}
\FloatTok{45621.0+2838.8}
\end{Highlighting}
\end{Shaded}

\begin{verbatim}
## [1] 48459.8
\end{verbatim}

\end{document}
